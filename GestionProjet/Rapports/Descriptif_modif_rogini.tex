\documentclass[12pt]{report} % format du document
\usepackage[utf8]{inputenc} % type d'encodage utf8 unix, le plus worlwide
\usepackage[round]{natbib} % package de format biblio
\usepackage[T1]{fontenc} % police d'encodage, OT1 moins de caractère que T1 (qui contient les caractères spéciaux)
\usepackage[english,francais]{babel} % package de langue
\usepackage{amsmath} % pour écrire les formules maths et améliorer la sortie
\usepackage{amssymb} % symboles additionnels pour amsmath
\usepackage{mathrsfs} % des symboles additionnels de physique (entre autre)
\usepackage[pdftex]{graphicx} % package pour les graphiques, version avancée
\usepackage[hmargin=2.5cm, vmargin=2cm, headheight=15.5pt]{geometry} % marges
\usepackage{setspace} % interlignes
\usepackage{fancyhdr} % customisation des hauts et bas de page
\usepackage[svgnames, table]{xcolor} % contrôle des couleurs avancé
\usepackage{listings} % pour du code, certains langages prédéfini, sortie simple
\usepackage{lastpage} % pour récupérer la dernière page
\usepackage{times} % police d'écriture
\usepackage{titlesec} % formatage des titres
\usepackage{framed}
%\overfullrule=2cm
\usepackage{eurosym}
\usepackage{setspace}
\usepackage[font=footnotesize]{caption}
\usepackage{multicol}
\usepackage{pdflscape}
\usepackage[unicode=true, pdfpagelabels]{hyperref} % pour les liens et customisation de ceux-ci
\usepackage{float}
\usepackage[font=footnotesize]{subcaption}
\usepackage[toc,page]{appendix}

\newfloat{Matrice}{h}{mat}

\DeclareUnicodeCharacter{20AC}{\euro}

\titlespacing{\paragraph}{0em}{1em}{1em} % pas de tabulation pour subsubsection, 1.5em espace au dessus, 0.5em en dessous

%--------------- Configuration couleurs & auteur du pdf ----------------------------------
\hypersetup{breaklinks=true,
            pdfauthor={Bastien Delseny\\
            Runghen Rogini},
            pdftitle={StatEasy},
            colorlinks=true,
            citecolor=DarkGreen,
            urlcolor=blue,
            linkcolor=DarkRed,
            pdfborder={0 0 0}}
%------------------------------------------------------------------------------------------


\urlstyle{same}  % don't use monospace font for urls
\newcommand{\HRule}{\rule{\linewidth}{0.5mm}} %Hrule fait des lignes horizontales de 0.5mm d'épaisseur

%------------------ On met une ligne en haut et bas de page -----------------------------
\renewcommand{\headrulewidth}{0.1pt}
%\renewcommand{\footrulewidth}{0.1pt}
%----------------------------------------------------------------------------------------

%----- On crée une commande pour faire des saut de paragraphe réglables 0em par défault ----------
\newcommand{\PAR}[1][0em]{\setlength{\parskip}{#1}\par}
%----------------------------------------------------------------------------------------

%------ Configuration des chapitres -----------------------------------------------------
\titleformat{\chapter}{\normalfont\Large\bfseries}{Chapitre \ \thechapter \space\space\space}{0ex}{} %titre et numéro chapitre sur la même ligne
\titlespacing{\chapter}{0em}{2em}{1em} % Espaces avant et après le chapitre (vertical)
%----------------------------------------------------------------------------------------
\makeatletter
\renewcommand\chapter{%\if@openright\cleardoublepage\else\clearpage\fi

	\thispagestyle{fancy}
		\lhead{}
		\chead{\leftmark}
		\rhead{}
		\lfoot{}
		\cfoot{}
		\rfoot{\thepage/\pageref*{ArabePage}}
                   \global\@topnum\z@
                    \@afterindentfalse
                    \secdef\@chapter\@schapter}
\makeatother
%----------------------------------------------------------------------------------------

\setcounter{tocdepth}{4} % Niveau de titres dans la table des matières

\AtBeginDocument{\addtocontents{toc}{\protect\thispagestyle{empty}}}
%-----------------------------------------------------------------------------------------
%-----------------------------------------------------------------------------------------
%-----------------------------------------------------------------------------------------






%%%%%%%%%%%%%%%%%%%%%%%%%%%%%%%%%%%%%%%%%%%%%%%%%%%%%%%%%%%%%%%%%%%%%%%%%%%%%%%%%%
%--------------------------------------------------------------------------------%
%				DOCUMENT														 %
%--------------------------------------------------------------------------------%
%%%%%%%%%%%%%%%%%%%%%%%%%%%%%%%%%%%%%%%%%%%%%%%%%%%%%%%%%%%%%%%%%%%%%%%%%%%%%%%%%%
\begin{document}

%--------------------------------------------------------------------------------
%			Page de garde						
%--------------------------------------------------------------------------------
\pagestyle{empty}

\title{StatEasy}
\author{Delseny Bastien\\
   Runghen Rogini}
\date{\today}

\maketitle

\clearpage

%--------------------------------------------------------------------------------
%			Table des matières					
%--------------------------------------------------------------------------------
\clearpage
\normalsize
\renewcommand{\contentsname}{Table des matières}
\tableofcontents
%\renewcommand{\listfigurename}{Images et figures}
%\listoffigures


%--------------------------------------------------------------------------------
%				RAPPORT						
%--------------------------------------------------------------------------------
\clearpage

	\pagestyle{fancy}
		\lhead{}
		\chead{\leftmark}
		\rhead{}
		\lfoot{}
		\cfoot{}
		\rfoot{\thepage/\pageref*{ArabePage}}


%--------------------------------------------------------------------------------
%			TEXTE DU RAPPORT					Banane
%--------------------------------------------------------------------------------
\setcounter{page}{1}
\setstretch{1.5}
\chapter{What is \emph{StatEasy} ?}

\emph{StatEasy} is a project which is meant to help easing statistical 
analysis for different users(varying from beginners to advanced users). 
We decided to divide this project into three sub-projects. 
It can actually be regarded as different aspects of our main goal 
- i.e. making the most of an open source tool for statistical analysis 
using a user-friendly interface. We talk about 'sub-projects' as being
various aspects of our project which will enable us to building up \emph{StatEasy}.

\\

\emph{StatEasy} aims to provide a user-friendly interface for R statistical analysis. 
Our main objectives are the following:
\begin{itemize}
\item Designing a user-friendly interface
\item Creating a database which will contain all available packages and related 
functions on R with respect to the different statistical analyses to be performed
\item Compiling links and tutorials on statistical analysis and R in general 
which are freely available. In addition to that, other documentation will be 
supplemented whereby information is currently lacking online. 
(All this will be in the format of a documentation available offline?)
\end{itemize}

\chapter{Why this tool ? - The purpose of our project}

Compared to available user-friendly environments such as Statisca or SPSS (and many others), 
we are suggesting using the open source tool - R statistical program (or software??). 
Even if R is becoming increasingly popular, a lot of people are still scared of it due to 
its unintuitive appearance. 
Like mentioned in an article in 2009 by Petchey and colleagues,
R is an advantageous tool which provides various analysis within only one program. 
However using R requires some training to getting used to the new work environment. 
\emph{StatEasy} can be viewed as a facilitator which enable different user types to 
get rid of their phobia of using R or their phobia of statistical analyses. 

For beginners who are not familiar to statistical analysis - we will be providing 
them some guidance to help them to choose the most appropriate analysis with respect 
to their objective. Currently, there are more than 10 000 packages available. 
Chances are high that the analysis that you are currently performing has been 
bundled in a nice package with all the required functions. 
Thus, to the more advanced users, \emph{StatEasy} will be 
used more like a search engine for the available packages they might be interested
to use wrt their analyses. 

We want to provide a tool which will need less time to be handled than R with 
R performances. Moreover \emph{StatEasy} also means less time learning statistical 
analysis, more time performing those.

\chapter{How to make all this possible ?}
As mentioned previously, we will be using R for our current project. 
We can perform statistical analysis, plus it is a useful tool 
for the management of spreadsheets and graphics. 
Being an open-source is of course an advantage.\\

To be able to carry out the different objectives of our project, 
we have separated the different objectives as distinct sub-projects. 
In our first phase, i.e. our current phase, of our project we are 
focusing on identifying the different type of users, the different 
kind of problems that are frequently encountered either from an R 
point of view or from a statistical point of view. For starters, we 
focused on the main questions that might help someone to identify the
statistical analysis they might be carrying out. 



%--------------------------------------------------------------------------------
%				BIBLIOGRAPHIE					
%--------------------------------------------------------------------------------
%\normalsize
%\phantomsection
%\setstretch{1}
%\addcontentsline{toc}{chapter}{R\'ef\'erences bibliographiques}
%\bibliographystyle{plainnat}
%\bibliography{biblio}
%\nocite{*}
\label{ArabePage}
\clearpage


\end{document}
%test
