\documentclass[12pt]{report} % format du document
\usepackage[utf8]{inputenc} % type d'encodage utf8 unix, le plus worlwide
\usepackage[round]{natbib} % package de format biblio
\usepackage[T1]{fontenc} % police d'encodage, OT1 moins de caractère que T1 (qui contient les caractères spéciaux)
\usepackage[english,francais]{babel} % package de langue
\usepackage{amsmath} % pour écrire les formules maths et améliorer la sortie
\usepackage{amssymb} % symboles additionnels pour amsmath
\usepackage{mathrsfs} % des symboles additionnels de physique (entre autre)
\usepackage[pdftex]{graphicx} % package pour les graphiques, version avancée
\usepackage[hmargin=2.5cm, vmargin=2cm, headheight=15.5pt]{geometry} % marges
\usepackage{setspace} % interlignes
\usepackage{fancyhdr} % customisation des hauts et bas de page
\usepackage[svgnames, table]{xcolor} % contrôle des couleurs avancé
\usepackage{listings} % pour du code, certains langages prédéfini, sortie simple
\usepackage{lastpage} % pour récupérer la dernière page
\usepackage{times} % police d'écriture
\usepackage{titlesec} % formatage des titres
\usepackage{framed}
%\overfullrule=2cm
\usepackage{eurosym}
\usepackage{setspace}
\usepackage[font=footnotesize]{caption}
\usepackage{multicol}
\usepackage{pdflscape}
\usepackage[unicode=true, pdfpagelabels]{hyperref} % pour les liens et customisation de ceux-ci
\usepackage{float}
\usepackage[font=footnotesize]{subcaption}
\usepackage[toc,page]{appendix}

\newfloat{Matrice}{h}{mat}

\DeclareUnicodeCharacter{20AC}{\euro}

\titlespacing{\paragraph}{0em}{1em}{1em} % pas de tabulation pour subsubsection, 1.5em espace au dessus, 0.5em en dessous

%--------------- Configuration couleurs & auteur du pdf ----------------------------------
\hypersetup{breaklinks=true,
            pdfauthor={Bastien Delseny\\
            Runghen Rogini},
            pdftitle={StatEasy},
            colorlinks=true,
            citecolor=DarkGreen,
            urlcolor=blue,
            linkcolor=DarkRed,
            pdfborder={0 0 0}}
%------------------------------------------------------------------------------------------


\urlstyle{same}  % don't use monospace font for urls
\newcommand{\HRule}{\rule{\linewidth}{0.5mm}} %Hrule fait des lignes horizontales de 0.5mm d'épaisseur

%------------------ On met une ligne en haut et bas de page -----------------------------
\renewcommand{\headrulewidth}{0.1pt}
%\renewcommand{\footrulewidth}{0.1pt}
%----------------------------------------------------------------------------------------

%----- On crée une commande pour faire des saut de paragraphe réglables 0em par défault ----------
\newcommand{\PAR}[1][0em]{\setlength{\parskip}{#1}\par}
%----------------------------------------------------------------------------------------

%------ Configuration des chapitres -----------------------------------------------------
\titleformat{\chapter}{\normalfont\Large\bfseries}{Chapitre \ \thechapter \space\space\space}{0ex}{} %titre et numéro chapitre sur la même ligne
\titlespacing{\chapter}{0em}{2em}{1em} % Espaces avant et après le chapitre (vertical)
%----------------------------------------------------------------------------------------
\makeatletter
\renewcommand\chapter{%\if@openright\cleardoublepage\else\clearpage\fi

	\thispagestyle{fancy}
		\lhead{}
		\chead{\leftmark}
		\rhead{}
		\lfoot{}
		\cfoot{}
		\rfoot{\thepage/\pageref*{ArabePage}}
                   \global\@topnum\z@
                    \@afterindentfalse
                    \secdef\@chapter\@schapter}
\makeatother
%----------------------------------------------------------------------------------------

\setcounter{tocdepth}{4} % Niveau de titres dans la table des matières

\AtBeginDocument{\addtocontents{toc}{\protect\thispagestyle{empty}}}
%-----------------------------------------------------------------------------------------
%-----------------------------------------------------------------------------------------
%-----------------------------------------------------------------------------------------






%%%%%%%%%%%%%%%%%%%%%%%%%%%%%%%%%%%%%%%%%%%%%%%%%%%%%%%%%%%%%%%%%%%%%%%%%%%%%%%%%%
%--------------------------------------------------------------------------------%
%				DOCUMENT														 %
%--------------------------------------------------------------------------------%
%%%%%%%%%%%%%%%%%%%%%%%%%%%%%%%%%%%%%%%%%%%%%%%%%%%%%%%%%%%%%%%%%%%%%%%%%%%%%%%%%%
\begin{document}

%--------------------------------------------------------------------------------
%			Page de garde						
%--------------------------------------------------------------------------------
\pagestyle{empty}

\title{StatEasy}
\author{Delseny Bastien\\
   Runghen Rogini}
\date{\today}

\maketitle

\clearpage

%--------------------------------------------------------------------------------
%			Table des matières					
%--------------------------------------------------------------------------------
\clearpage
\normalsize
\renewcommand{\contentsname}{Table des matières}
\tableofcontents
%\renewcommand{\listfigurename}{Images et figures}
%\listoffigures


%--------------------------------------------------------------------------------
%				RAPPORT						
%--------------------------------------------------------------------------------
\clearpage

	\pagestyle{fancy}
		\lhead{}
		\chead{\leftmark}
		\rhead{}
		\lfoot{}
		\cfoot{}
		\rfoot{\thepage/\pageref*{ArabePage}}


%--------------------------------------------------------------------------------
%			TEXTE DU RAPPORT					
%--------------------------------------------------------------------------------
\setcounter{page}{1}
\setstretch{1.5}
\chapter{What is \emph{StatEasy} ?}
\emph{StatEasy} is a project which is meant to help performing statistical analysis.
In the actual state \emph{StatEasy} our goal is to divide this project into three sub-projects.
The first sub-project is also its main goal. 
\emph{StatEasy} will be developed to provide a friendly-user interface for R statistical analysis.
The second sub-project of \emph{StatEasy} is providing an easy to use database of all packages and functions existing on R.
This database is meant to be accessible via the interface and to provide a powerfull and usefull search engine among the large amount of functions existing on R.
The third but not least part of \emph{StatEasy} project is developing a powerfull documentation for statistical analysis and R use.
This documentation is meant to provide help and documentation on a large amount of statistical analysis to all people who will need.
This documentation will regroup links and resources for statistical analysis and R tutorial.
It will also focus on providing tutorials where there are lacks in R and statistical community.

\chapter{Why this tool ?}
This tool is meant to help people performing good statistical analysis.
While R exist to perform good statistical analysis in a technical term \emph{StatEasy} is a tool which means to choose the right analysis.
Studies are taking more and more times as researches.
We want to provide a tool which will need less time to be handled than R with R performances.
Moreover \emph{StatEasy} also means less time learning statistical analysis, more time performing those.

\chapter{How to make all this possible ?}
It already exist a powerfull tool for performing statistical analysis which provide useful tools for spreadsheet management and graphics.
This tool is R and is open-source.
But R is difficult to handle and it can be difficult to find the tool you need on it.
\emph{StatEasy} will be an interface meaning to be user-friendly but as powerfull as R.
One way to do so is to develop this interface for R.
By doing so we will be able to provide a powerfull tool in technical terms with all its advantages and more.
\emph{StatEasy} is also meant to be handled at different scales, from statistical expert with competences of R programmations to new to statistics and programmation.
We are now targeting three different scales of use.
The first one is providing a powerfull and easy to handle database of R packages and functions for statistical experts who wants to find the tool they need in a fast and easy way.
The second one is having an interactive tool that helps people choosing the right statistical analysis and how to perform it.
The last one is providing an automatical analysis of data.

%--------------------------------------------------------------------------------
%				BIBLIOGRAPHIE					
%--------------------------------------------------------------------------------
%\normalsize
%\phantomsection
%\setstretch{1}
%\addcontentsline{toc}{chapter}{R\'ef\'erences bibliographiques}
%\bibliographystyle{plainnat}
%\bibliography{biblio}
%\nocite{*}
\label{ArabePage}
\clearpage


\end{document}
%test
